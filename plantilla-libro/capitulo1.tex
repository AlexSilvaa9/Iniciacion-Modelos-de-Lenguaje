\chapter{Programación concurrente en Java}
\chapterauthor{Alberto G. Salguero Hidalgo}

\section{Introducción}

Todas las imágenes deben de estar referenciadas desde el texto. Para ello puede usarse el comando \textbackslash ref\{etiqueta\}. Este es un ejemplo que hace referencia a la figura \ref{fig:etiqueta1}. Esto también se aplica a las tablas (\ref{demo-table}) y los fragmentos de código (\ref{fig:ejemplo-java}). Para citar trabajos conviene usar el comando \textbackslash cite{etiqueta}. Puede obtener más información en \cite{greenwade93} \cite{goossens93}. La referencias hay que declararlas previamente en el archivo \textit{bibliography.bib}.

La programación concurrente en Java se refiere a la capacidad de un programa para realizar varias tareas simultáneamente, ya sea en un solo procesador o en varios procesadores que trabajan en paralelo. La programación concurrente se utiliza a menudo en aplicaciones en tiempo real, como sistemas operativos, bases de datos, sistemas de comunicaciones y aplicaciones web.

La programación concurrente en Java se refiere a la capacidad de un programa para realizar varias tareas simultáneamente, ya sea en un solo procesador o en varios procesadores que trabajan en paralelo. La programación concurrente se utiliza a menudo en aplicaciones en tiempo real, como sistemas operativos, bases de datos, sistemas de comunicaciones y aplicaciones web. 

\begin{figure}
	\centering
	\includegraphics[width=0.25\linewidth]{img/tex-icon}
	\caption{Ejemplo de figura.}
	\label{fig:etiqueta1}
\end{figure}

\section{Sección intermedia}

La programación concurrente en Java se refiere a la capacidad de un programa para realizar varias tareas simultáneamente, ya sea en un solo procesador o en varios procesadores que trabajan en paralelo. La programación concurrente se utiliza a menudo en aplicaciones en tiempo real, como sistemas operativos, bases de datos, sistemas de comunicaciones y aplicaciones web.

\begin{figure}[h]
	\centering
	\begin{lstlisting}[language=Java, style=ColorStyle]
public class HelloWorld {
	public static void main(String[] args) {
		System.out.println("Hola, mundo!"); 
	}
}
	\end{lstlisting}
	\caption{Ejemplo de código Java}
	\label{fig:ejemplo-java}
\end{figure}

La programación concurrente en Java se refiere a la capacidad de un programa para realizar varias tareas simultáneamente, ya sea en un solo procesador o en varios procesadores que trabajan en paralelo. La programación concurrente se utiliza a menudo en aplicaciones en tiempo real, como sistemas operativos, bases de datos, sistemas de comunicaciones y aplicaciones web.

\begin{table}
	\begin{center}
		\begin{tabular}{c r r}
			\hline
			& Peso & Altura \\
			\hline
			Ana & 62 & 165 \\
			Fran & 84 & 183 \\
			Ana & 62 & 165 \\
			Fran & 84 & 183 \\
			\hline
		\end{tabular}
		\caption{Ejemplo de tabla.}
		\label{demo-table}
	\end{center}
\end{table}

La programación concurrente en Java se refiere a la capacidad de un programa para realizar varias tareas simultáneamente, ya sea en un solo procesador o en varios procesadores que trabajan en paralelo. La programación concurrente se utiliza a menudo en aplicaciones en tiempo real, como sistemas operativos, bases de datos, sistemas de comunicaciones y aplicaciones web.

La programación concurrente en Java se refiere a la capacidad de un programa para realizar varias tareas simultáneamente, ya sea en un solo procesador o en varios procesadores que trabajan en paralelo. La programación concurrente se utiliza a menudo en aplicaciones en tiempo real, como sistemas operativos, bases de datos, sistemas de comunicaciones y aplicaciones web.

\section{Conclusiones}

La programación concurrente en Java se refiere a la capacidad de un programa para realizar varias tareas simultáneamente, ya sea en un solo procesador o en varios procesadores que trabajan en paralelo. La programación concurrente se utiliza a menudo en aplicaciones en tiempo real, como sistemas operativos, bases de datos, sistemas de comunicaciones y aplicaciones web.

\bibliography{biblio1}
