\documentclass{book}
\usepackage{graphicx}
\usepackage{color}
\usepackage{tocloft}
\usepackage{titlesec}
\usepackage{xcolor}
\usepackage[spanish]{babel}
\usepackage{fancyhdr}\usepackage{geometry}
\usepackage{listings}
\usepackage{authblk}
\usepackage{hyperref}
\usepackage[pages=some]{background}
\usepackage[sectionbib]{natbib}
\usepackage{chapterbib}

\bibliographystyle{dcu}

% solo el tículo del capítulo en el índice
\setcounter{tocdepth}{0}

\backgroundsetup{
	scale=8,
	color=black,
	opacity=0.2,
	angle=0,
	position=current page.south east,nodeanchor=south east,
	hshift=1.0in, vshift=-0.8in,
	contents={%
		%\includegraphics[width=0.5\paperwidth,height=0.25\paperheight]{lib/logos/uma-positiva-vertical}
		\includegraphics{img/uma-positiva-vertical}
	}%
}

\hypersetup{
	colorlinks=true,
	linkcolor=blue,
	filecolor=magenta,      
	urlcolor=cyan,
	pdftitle={Overleaf Example},
	pdfpagemode=FullScreen,
}

\lstdefinestyle{ColorStyle}{
	basicstyle=\ttfamily\color{black},
	keywordstyle=\color{magenta},
	commentstyle=\color{green},
	stringstyle=\color{red},
	numbers=left,
	frame=single
}

\makeatletter
\newcommand{\chapterauthor}[1]{%
	{\parindent0pt\vspace*{-25pt}%
		\linespread{1.1}\large\scshape#1%
		\par\nobreak\vspace*{35pt}}
	\@afterheading%
}
\makeatother

% Establece los márgenes de los capítulos a 2 cm a la izquierda y 1 cm en el resto de los lados
%\titleformat{\chapter}[display]
%{\newgeometry{left=2cm,right=1cm,top=1cm,bottom=1cm}\normalfont\huge\bfseries}
%{\filleft\chaptertitlename\ \thechapter}
%{20pt}
%{\Huge\color{white}\vspace{1cm}\filright}
%[\restoregeometry] % Restaura los márgenes originales después del título del capítulo


% Define el color del encabezado
\definecolor{myblue}{RGB}{24, 50, 86}

% Define el formato del encabezado en las páginas de inicio de cada capítulo
%\titleformat{\chapter}[display]
%{\normalfont\huge\bfseries}{\filleft\chaptertitlename\ \thechapter}{20pt}{\Huge\color{white}\vspace{1cm}\filright}

\titleformat{\chapter}[frame]
{\normalfont}  % formato
{\filleft\normalfont\huge\bfseries \ \chaptertitlename\ \thechapter \ }  % etiqueta
{30pt}  % separación
{\Large\bfseries}  % código anterior
{} % código posterior

% Define el formato de la etiqueta en el índice
\renewcommand{\cftchapleader}{\cftdotfill{\cftdotsep}}

% Define el formato del encabezado en las demás páginas
\pagestyle{headings}
%\renewcommand{\chaptermark}[1]{\markboth{\chaptername\ \thechapter.\ #1.}{}}
\renewcommand{\chaptermark}[1]{\markboth{\chaptername\ \thechapter.\ #1.}{}}
\renewcommand{\sectionmark}[1]{\markright{\thesection.\ #1}}

% Define el formato del encabezado en las páginas de inicio de cada capítulo
%\fancypagestyle{plain}{
%	\fancyhf{}
%	\renewcommand{\headrulewidth}{0pt}
%	\renewcommand{\footrulewidth}{0pt}
%	\fancyhead[C]{\colorbox{myblue}{\makebox[\dimexpr\linewidth-2\fboxsep][l]{\Huge\sffamily\color{white}\ \textbf{\chaptername\ \thechapter}}}}
%}

\begin{document}
	\renewcommand{\tablename}{Tabla}
	
	% Portada
	\BgThispage
	\thispagestyle{empty}
	
	\begin{center}
		\includegraphics[width=\textwidth]{img/logo1.png}
		\vspace{2cm}
		
		{\Large\textbf{Trabajos de la asignatura}\\}
		{\Huge\textbf{Programación Avanzada en Bionformática}}
		\vspace{1cm}
		
		{\Large\textbf{Grado en Ingeniería de la Salud}}
		
		\vspace{1.5cm}
		
		{\Large\textbf{Curso 2012-13}}
		
		\vfill
		
		\begin{minipage}{0.4\textwidth}
			\begin{center}
				\includegraphics[width=\textwidth]{img/etsii}
			\end{center}
		\end{minipage}
		\begin{minipage}{0.4\textwidth}
			\begin{center}
%				\textbf{Facultad de Ciencias}\\
%				\textbf{Universidad de Málaga}\\
%				\textbf{Año}
				\includegraphics[width=.6\textwidth]{img/lcc}
			\end{center}
		\end{minipage}
	\end{center}
	
	\newpage
	
	\chapter*{Prólogo}
	
	Este libro, editado por el Prof. Alberto G. Salguero Hidalgo, contiene los trabajos realizados por los alumnos de la asignatura Programación Avanzada en Bioinformática, del Grado en Ingeniería de la Salud de la Universidad de Málaga. 
	
	Los alumnos son plenos responsables de su contribución al libro y conservan todos los derechos de autoría del contenido de sus respectivos capítulos.
	
	\newpage
	
	% Índice
	\tableofcontents
	\thispagestyle{empty} % Oculta el número de página en la página del índice
	
	\newpage
	
	% Contenido del libro
	\chapter{Programación concurrente en Java}
\chapterauthor{Alberto G. Salguero Hidalgo}

\section{Introducción}

Todas las imágenes deben de estar referenciadas desde el texto. Para ello puede usarse el comando \textbackslash ref\{etiqueta\}. Este es un ejemplo que hace referencia a la figura \ref{fig:etiqueta1}. Esto también se aplica a las tablas (\ref{demo-table}) y los fragmentos de código (\ref{fig:ejemplo-java}). Para citar trabajos conviene usar el comando \textbackslash cite{etiqueta}. Puede obtener más información en \cite{greenwade93} \cite{goossens93}. La referencias hay que declararlas previamente en el archivo \textit{bibliography.bib}.

La programación concurrente en Java se refiere a la capacidad de un programa para realizar varias tareas simultáneamente, ya sea en un solo procesador o en varios procesadores que trabajan en paralelo. La programación concurrente se utiliza a menudo en aplicaciones en tiempo real, como sistemas operativos, bases de datos, sistemas de comunicaciones y aplicaciones web.

La programación concurrente en Java se refiere a la capacidad de un programa para realizar varias tareas simultáneamente, ya sea en un solo procesador o en varios procesadores que trabajan en paralelo. La programación concurrente se utiliza a menudo en aplicaciones en tiempo real, como sistemas operativos, bases de datos, sistemas de comunicaciones y aplicaciones web. 

\begin{figure}
	\centering
	\includegraphics[width=0.25\linewidth]{img/tex-icon}
	\caption{Ejemplo de figura.}
	\label{fig:etiqueta1}
\end{figure}

\section{Sección intermedia}

La programación concurrente en Java se refiere a la capacidad de un programa para realizar varias tareas simultáneamente, ya sea en un solo procesador o en varios procesadores que trabajan en paralelo. La programación concurrente se utiliza a menudo en aplicaciones en tiempo real, como sistemas operativos, bases de datos, sistemas de comunicaciones y aplicaciones web.

\begin{figure}[h]
	\centering
	\begin{lstlisting}[language=Java, style=ColorStyle]
public class HelloWorld {
	public static void main(String[] args) {
		System.out.println("Hola, mundo!"); 
	}
}
	\end{lstlisting}
	\caption{Ejemplo de código Java}
	\label{fig:ejemplo-java}
\end{figure}

La programación concurrente en Java se refiere a la capacidad de un programa para realizar varias tareas simultáneamente, ya sea en un solo procesador o en varios procesadores que trabajan en paralelo. La programación concurrente se utiliza a menudo en aplicaciones en tiempo real, como sistemas operativos, bases de datos, sistemas de comunicaciones y aplicaciones web.

\begin{table}
	\begin{center}
		\begin{tabular}{c r r}
			\hline
			& Peso & Altura \\
			\hline
			Ana & 62 & 165 \\
			Fran & 84 & 183 \\
			Ana & 62 & 165 \\
			Fran & 84 & 183 \\
			\hline
		\end{tabular}
		\caption{Ejemplo de tabla.}
		\label{demo-table}
	\end{center}
\end{table}

La programación concurrente en Java se refiere a la capacidad de un programa para realizar varias tareas simultáneamente, ya sea en un solo procesador o en varios procesadores que trabajan en paralelo. La programación concurrente se utiliza a menudo en aplicaciones en tiempo real, como sistemas operativos, bases de datos, sistemas de comunicaciones y aplicaciones web.

La programación concurrente en Java se refiere a la capacidad de un programa para realizar varias tareas simultáneamente, ya sea en un solo procesador o en varios procesadores que trabajan en paralelo. La programación concurrente se utiliza a menudo en aplicaciones en tiempo real, como sistemas operativos, bases de datos, sistemas de comunicaciones y aplicaciones web.

\section{Conclusiones}

La programación concurrente en Java se refiere a la capacidad de un programa para realizar varias tareas simultáneamente, ya sea en un solo procesador o en varios procesadores que trabajan en paralelo. La programación concurrente se utiliza a menudo en aplicaciones en tiempo real, como sistemas operativos, bases de datos, sistemas de comunicaciones y aplicaciones web.

\bibliography{biblio1}

	
	%\chapter{Programación concurrente en Java}
\chapterauthor{Alberto G. Salguero Hidalgo}

\section{Introducción}

Todas las imágenes deben de estar referenciadas desde el texto. Para ello puede usarse el comando \textbackslash ref\{etiqueta\}. Este es un ejemplo que hace referencia a la figura \ref{fig:etiqueta1}. Esto también se aplica a las tablas (\ref{demo-table}) y los fragmentos de código (\ref{fig:ejemplo-java}). Para citar trabajos conviene usar el comando \textbackslash cite{etiqueta}. Puede obtener más información en \cite{greenwade93} \cite{goossens93}. La referencias hay que declararlas previamente en el archivo \textit{bibliography.bib}.

La programación concurrente en Java se refiere a la capacidad de un programa para realizar varias tareas simultáneamente, ya sea en un solo procesador o en varios procesadores que trabajan en paralelo. La programación concurrente se utiliza a menudo en aplicaciones en tiempo real, como sistemas operativos, bases de datos, sistemas de comunicaciones y aplicaciones web.

La programación concurrente en Java se refiere a la capacidad de un programa para realizar varias tareas simultáneamente, ya sea en un solo procesador o en varios procesadores que trabajan en paralelo. La programación concurrente se utiliza a menudo en aplicaciones en tiempo real, como sistemas operativos, bases de datos, sistemas de comunicaciones y aplicaciones web. 

\begin{figure}
	\centering
	\includegraphics[width=0.25\linewidth]{img/tex-icon}
	\caption{Ejemplo de figura.}
	\label{fig:etiqueta1}
\end{figure}

\section{Sección intermedia}

La programación concurrente en Java se refiere a la capacidad de un programa para realizar varias tareas simultáneamente, ya sea en un solo procesador o en varios procesadores que trabajan en paralelo. La programación concurrente se utiliza a menudo en aplicaciones en tiempo real, como sistemas operativos, bases de datos, sistemas de comunicaciones y aplicaciones web.

\begin{figure}[h]
	\centering
	\begin{lstlisting}[language=Java, style=ColorStyle]
public class HelloWorld {
	public static void main(String[] args) {
		System.out.println("Hola, mundo!"); 
	}
}
	\end{lstlisting}
	\caption{Ejemplo de código Java}
	\label{fig:ejemplo-java}
\end{figure}

La programación concurrente en Java se refiere a la capacidad de un programa para realizar varias tareas simultáneamente, ya sea en un solo procesador o en varios procesadores que trabajan en paralelo. La programación concurrente se utiliza a menudo en aplicaciones en tiempo real, como sistemas operativos, bases de datos, sistemas de comunicaciones y aplicaciones web.

\begin{table}
	\begin{center}
		\begin{tabular}{c r r}
			\hline
			& Peso & Altura \\
			\hline
			Ana & 62 & 165 \\
			Fran & 84 & 183 \\
			Ana & 62 & 165 \\
			Fran & 84 & 183 \\
			\hline
		\end{tabular}
		\caption{Ejemplo de tabla.}
		\label{demo-table}
	\end{center}
\end{table}

La programación concurrente en Java se refiere a la capacidad de un programa para realizar varias tareas simultáneamente, ya sea en un solo procesador o en varios procesadores que trabajan en paralelo. La programación concurrente se utiliza a menudo en aplicaciones en tiempo real, como sistemas operativos, bases de datos, sistemas de comunicaciones y aplicaciones web.

La programación concurrente en Java se refiere a la capacidad de un programa para realizar varias tareas simultáneamente, ya sea en un solo procesador o en varios procesadores que trabajan en paralelo. La programación concurrente se utiliza a menudo en aplicaciones en tiempo real, como sistemas operativos, bases de datos, sistemas de comunicaciones y aplicaciones web.

\section{Conclusiones}

La programación concurrente en Java se refiere a la capacidad de un programa para realizar varias tareas simultáneamente, ya sea en un solo procesador o en varios procesadores que trabajan en paralelo. La programación concurrente se utiliza a menudo en aplicaciones en tiempo real, como sistemas operativos, bases de datos, sistemas de comunicaciones y aplicaciones web.

\bibliography{biblio1}
  % se puede borrar para que no salga duplicado.

	
\end{document}
